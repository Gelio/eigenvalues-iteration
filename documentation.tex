\documentclass[12pt]{article}
\usepackage{polski}
\usepackage[utf8]{inputenc}
\usepackage{amsfonts}
\usepackage{amsmath}
\usepackage{enumitem}
\usepackage{graphicx}
\usepackage{float}
\usepackage{centernot}
\setlength{\parskip}{1em}


\begin{document}
	\title{Sprawozdanie\\Metody Numeryczne 2, laboratorium 5}
	\author{Grzegorz Rozdzialik (D4, grupa lab. 2)}
	\maketitle	
	
	\section{Zadanie}
	{\Large Temat \textbf{5}, zadanie \textbf{22}:}\\
	Stosując metodę potęgową (z normowaniem) oraz deflację $A_1 = A - \lambda x x^*$ oblicz wszystkie wartości własne macierzy trójdiagonalnej $A$, gdzie $a_{k,k} = 5, a_{k, k - 1} = 2+i, a_{k, k+1} = 2-i$.
	
	Niech $n \in \mathbb{N}$. Macierz $A \in \mathbb{C}^{n \times n}$ ma postać:
	\begin{equation*}
		A =
		\begin{bmatrix}
			5      & 2 - i  & 0      & 0      & \dots  & 0      & 0      & 0      \\
			2+i    & 5      & 2-i    & 0      & \dots  & 0      & 0      & 0      \\
			0      & 2+i    & 5      & 2-i    & \dots  & 0      & 0      & 0      \\
			0      & 0      & 2+i    & 5      & \dots  & 0      & 0      & 0      \\
			\vdots & \vdots & \vdots & \vdots & \ddots & \vdots & \vdots & \vdots \\
			0      & 0      & 0      & 0      & \dots  & 5      & 2-i    & 0      \\
			0      & 0      & 0      & 0      & \dots  & 2+i    & 5      & 2-i    \\
			0      & 0      & 0      & 0      & \dots  & 0      & 2+i    & 5
		\end{bmatrix}
	\end{equation*}
	Łatwo zauważyć, że macierz $A$ jest diagonalnie silnie dominująca kolumnowo i wierszowo.
	
	Należy znaleźć wszystkie własności własne $\lambda$ macierzy $A$, czyli takie, że:
	$$\exists x \in (\mathbb{C} \setminus 0) \hspace{10px}  Ax = \lambda x$$
	
	Wartości własne $\lambda$ są pierwiastkami wielomianu charakterystycznego macierzy A:
	$$X(\lambda) = \det (A - \lambda I)$$
	
	Z podstawowego twierdzenia algebry wiemy, że pierwiastków wielomianu $X(\lambda)$ jest $n$, co do krotności.
	
	
	\section{Opis metody}
	Niech $A_0 = A$.
	
	Ponieważ metoda potęgowa znajduje dominującą wartość własną macierzy $A_i$ ($i = 0, \dots, n-1$), należy zastosować deflację. Po znalezieniu dominującej wartości własnej $\lambda_i$ macierzy $A_i$ oraz wektora własnego $x_i \in \mathbb{C}^n$ związanego z wartością własną $\lambda_i$ macierz $A_{i+1}$ otrzymujemy w następujący sposób:
	$$ A_{i+1} = A_i - \lambda_i x_i x_i^* $$
	
	Wartość własna $\lambda_{i+1}$ macierzy $A_{i+1}$ jest mniejsza co do modułu od wartości własnej $\lambda_i$ ($|\lambda_{i+1}| < |\lambda_i|$), więc metoda potęgowa zastosowana na macierzy $A_{i+1}$ znajdzie wartość własną $\lambda_{i+1}$.
	
	Niech $\delta \in \mathbb{R}$ będzie zadaną dokładnością przybliżenia wartości własnej $\lambda_i$. Mając przybliżenie początkowej $x_i^{(0)} \in \mathbb{C}^n$ wektora własnego odpowiadającego wartości własnej $\lambda_i$ mamy:
	\begin{align}
		\begin{split}
			\label{przyblizenie-wektora}
			y_i^{(k)}       & = A_i x_i^{(k-1)} \\
			x_i^{(k)}       & = \frac{y_i^{(k)}}{|| y_i^{(k)} ||_2}
		\end{split}
	\end{align}

	\begin{align}
		\label{przyblizenie-wartosci}
		\lambda_i^{(k)} = \langle y_i^{(k)}, x_i^{(k-1)} \rangle
	\end{align}
	gdzie równania \eqref{przyblizenie-wektora} przybliżają wektor początkowy związany z dominującą wartością własną macierzy $A_i$, a równanie \eqref{przyblizenie-wartosci} przybliża tą dominującą wartość własną.
	
	Przybliżanie kontynuujemy, aż zostanie spełniony warunek stopu:
	$$ | \lambda_i^{(k)} - \lambda_i^{(k-1)} | < \delta $$
	
	Obliczenia powtarzamy dla $i = 0, \dots, n-1$.
	
	Po znalezieniu wszystkich wartości i wektorów własnych można sprawdzić jak dokładne obliczone przybliżenia. W tym celu zdefiniujmy:
	$$e_i = A x_i - \lambda_i x_i \hspace{30px} i = 0, 1, \dots, n-1$$
	\begin{equation*}
		E = 
		\begin{bmatrix}
			e_0 & e_1 & e_2 & \dots & e_{n-1}
		\end{bmatrix}
	\end{equation*}
	
	Licząc normę z macierzy $E$ otrzymujemy konkretną wartość oznaczającą dokładność znalezionych wartości i wektorów własnych.
	
	\section{Implementacja metody}
	Implementacja metody podzielona jest na następujące kroki:
	\begin{enumerate}
		\item Konstruowanie macierzy $A$ rozmiaru $n \times n$ (funkcja \texttt{constructMatrix})
		\item Wykonanie metody potęgowej z normowaniem w celu znalezienia dominującej wartości własnej $\lambda_i$ macierzy $A_i$ (funkcja \texttt{powerIteration})
		\item Powtarzanie kroku związanego z metodą potęgową oraz deflacja po znalezieniu $\lambda_i$ w celu znalezienia wszystkich wartości własnych macierzy $A$ (funkcja \texttt{findEigenvaluesAndVectors})
		\item Obliczenie macierzy błędu $E$ (funkcja \texttt{calculateErrorMatrix})
	\end{enumerate}

	
	
	\section{Poprawność metody}
	Biorąc wektor $x_i^{(0)} = [1 1 1 \dots 1]^T$ za przybliżenie początkowe metoda jest poprawna, ponieważ wektor ten ma każdą niezerową składową, zatem daje się przybliżyć do dowolnego wektora w przestrzeni wektorów własnych.
	
	Po wykonaniu testów stwierdzam, że metoda jest poprawna, znajduje wszystkie wartości własne macierzy $A$ oraz odpowiadające im wektory własne, a jej wyniki są zbliżone do tych z funkcji \texttt{eig} realizującej to samo zadanie, dostępnej w Matlabie.
	
	
	
	\section{Przykłady}
	Z powodu niewielkiej liczby parametrów w zadaniu przykłady pokazują porównanie metody potęgowej z normowaniem oraz metody \texttt{eig} dostępnej w Matlabie.
	
	W każdym przykładzie przyjęty jest limit iteracji równy $100$, ale w większości przypadków nie został on osiągnięty.
	
	\begin{enumerate}[label=\textbf{Przykład \arabic*}]
		\item
		$n = 5, \delta = 0.1$
		\begin{align*}
		\begin{tabular}{|c|c|c|}
			\hline
			                                        & metoda potęgowa & funkcja \texttt{eig} \\ \hline
			maksymalny błąd $A x_i - \lambda_i x_i$ &      16.0       &         14.7         \\ \hline
			       norma macierzy błędu $E$         &      22.2       &         22.5         \\ \hline
			           czas przybliżania            &    0.4330 ms    &      0.0912 ms       \\ \hline
		\end{tabular}
		\end{align*}
	
		Pod względem dokładności obie metody są podobne, jednak funkcja \texttt{eig} dominuje pod względem szybkości - jest ponad czterokrotnie szybsza.
		
		\item
		$n = 5, \delta = 10^{-5}$
		
		Zwiększenie dokładności spowodowało zmniejszenie maksymalnego błędu $A x_i - \lambda_i x_i$ metody potęgowej do 14.7 (wynik podobny do funkcji \texttt{eig}) przy jednoczesnym wydłużeniu czasu przybliżania do $2.61$ ms (ponad sześciokrotny wzrost).
		
		
		\item
		$n = 20, \delta = 0.1$
		\begin{align*}
		\begin{tabular}{|c|c|c|}
			\hline
			                                        & metoda potęgowa & funkcja \texttt{eig} \\ \hline
			maksymalny błąd $A x_i - \lambda_i x_i$ &      39.0       &         29.6         \\ \hline
			       norma macierzy błędu $E$         &      108.7      &        110.2         \\ \hline
			           czas przybliżania            &    6.7451 ms    &      0.9639 ms       \\ \hline
		\end{tabular}
		\end{align*}
		
		Zwiększenie rozmiaru macierzy zmniejszyło dokładność obu metod oraz wydłużyło ich czas działania.
		
	\end{enumerate}
	
	
	
	
	\section{Wnioski}
	\begin{enumerate}
		\item \textbf{TODO}
	\end{enumerate}
	
	
	
	\section{Funkcja do testowania metody}
	\textbf{TODO}
	
	
	
	\section{Bibliografia}
	\begin{enumerate}
		\item Informacje z wykładu \textit{Metod numerycznych 2} (wydział MiNI PW, dr Iwona Wróbel), w szczególności temat dotyczący \textit{metody potęgowej z normowaniem}, wraz z algorytmem.
	\end{enumerate}
	
\end{document}