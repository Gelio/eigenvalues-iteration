\documentclass[12pt]{article}
\usepackage{polski}
\usepackage[utf8]{inputenc}
\usepackage{amsfonts}
\usepackage{amsmath}
\usepackage{enumitem}
\usepackage{graphicx}
\usepackage{float}
\usepackage{centernot}
\setlength{\parskip}{1em}


\begin{document}
	\title{Sprawozdanie\\Metody Numeryczne 2, laboratorium 5}
	\author{Grzegorz Rozdzialik (D4, grupa lab. 2)}
	\maketitle	
	
	\section{Zadanie}
	{\Large Temat \textbf{5}, zadanie \textbf{22}:}\\
	Stosując metodę potęgową (z normowaniem) oraz deflację $A_1 = A - \lambda x x^*$ oblicz wszystkie wartości własne macierzy trójdiagonalnej $A$, gdzie $a_{k,k} = 5, a_{k, k - 1} = 2+i, a_{k, k+1} = 2-i$.
	
	Niech $n \in \mathbb{N}$. Macierz $A \in \mathbb{C}^{n \times n}$ ma postać:
	\begin{equation*}
		A =
		\begin{bmatrix}
			5      & 2 - i  & 0      & 0      & \dots  & 0      & 0      & 0      \\
			2+i    & 5      & 2-i    & 0      & \dots  & 0      & 0      & 0      \\
			0      & 2+i    & 5      & 2-i    & \dots  & 0      & 0      & 0      \\
			0      & 0      & 2+i    & 5      & \dots  & 0      & 0      & 0      \\
			\vdots & \vdots & \vdots & \vdots & \ddots & \vdots & \vdots & \vdots \\
			0      & 0      & 0      & 0      & \dots  & 5      & 2-i    & 0      \\
			0      & 0      & 0      & 0      & \dots  & 2+i    & 5      & 2-i    \\
			0      & 0      & 0      & 0      & \dots  & 0      & 2+i    & 5
		\end{bmatrix}
	\end{equation*}
	Łatwo zauważyć, że macierz $A$ jest diagonalnie silnie dominująca kolumnowo i wierszowo.
	
	Należy znaleźć wszystkie własności własne $\lambda$ macierzy $A$, czyli takie, że:
	$$\exists x \in (\mathbb{C} \setminus 0) \hspace{10px}  Ax = \lambda x$$
	
	Wartości własne $\lambda$ są pierwiastkami wielomianu charakterystycznego macierzy A:
	$$X(\lambda) = \det (A - \lambda I)$$
	
	Z podstawowego twierdzenia algebry wiemy, że pierwiastków wielomianu $X(\lambda)$ jest $n$, co do krotności.
	
	
	\section{Opis metody}
	Niech $A_0 = A$.
	
	Ponieważ metoda potęgowa znajduje dominującą wartość własną macierzy $A_i$ ($i = 0, \dots, n-1$), należy zastosować deflację. Po znalezieniu dominującej wartości własnej $\lambda_i$ macierzy $A_i$ oraz wektora własnego $x_i \in \mathbb{C}^n$ związanego z wartością własną $\lambda_i$ macierz $A_{i+1}$ otrzymujemy w następujący sposób:
	$$ A_{i+1} = A_i - \lambda_i x_i x_i^* $$
	
	Wartość własna $\lambda_{i+1}$ macierzy $A_{i+1}$ jest mniejsza co do modułu od wartości własnej $\lambda_i$ ($|\lambda_{i+1}| < |\lambda_i|$), więc metoda potęgowa zastosowana na macierzy $A_{i+1}$ znajdzie wartość własną $\lambda_{i+1}$.
	
	Niech $\delta \in \mathbb{R}$ będzie zadaną dokładnością przybliżenia wartości własnej $\lambda_i$. Mając przybliżenie początkowej $x_i^{(0)} \in \mathbb{C}^n$ wektora własnego odpowiadającego wartości własnej $\lambda_i$ mamy:
	\begin{align}
		\begin{split}
			\label{przyblizenie-wektora}
			y_i^{(k)}       & = A_i x_i^{(k-1)} \\
			x_i^{(k)}       & = \frac{y_i^{(k)}}{|| y_i^{(k)} ||_2}
		\end{split}
	\end{align}

	\begin{align}
		\label{przyblizenie-wartosci}
		\lambda_i^{(k)} = \langle y_i^{(k)}, x_i^{(k-1)} \rangle
	\end{align}
	gdzie równania \eqref{przyblizenie-wektora} przybliżają wektor początkowy związany z dominującą wartością własną macierzy $A_i$, a równanie \eqref{przyblizenie-wartosci} przybliża tą dominującą wartość własną.
	
	Przybliżanie kontynuujemy, aż zostanie spełniony warunek stopu:
	$$ | \lambda_i^{(k)} - \lambda_i^{(k-1)} | < \delta $$
	
	Obliczenia powtarzamy dla $i = 0, \dots, n-1$.
	
	Po znalezieniu wszystkich wartości i wektorów własnych można sprawdzić jak dokładne obliczone przybliżenia. W tym celu zdefiniujmy:
	$$e_i = A x_i - \lambda_i x_i \hspace{30px} i = 0, 1, \dots, n-1$$
	\begin{equation*}
		E = 
		\begin{bmatrix}
			e_0 & e_1 & e_2 & \dots & e_{n-1}
		\end{bmatrix}
	\end{equation*}
	
	Licząc normę z macierzy $E$ otrzymujemy konkretną wartość oznaczającą dokładność znalezionych wartości i wektorów własnych.
	
	\section{Implementacja metody}
	Implementacja metody podzielona jest na następujące kroki:
	\begin{enumerate}
		\item Konstruowanie macierzy $A$ rozmiaru $n \times n$ (funkcja \texttt{constructMatrix})
		\item Wykonanie metody potęgowej z normowaniem w celu znalezienia dominującej wartości własnej $\lambda_i$ macierzy $A_i$ (funkcja \texttt{powerIteration})
		\item Powtarzanie kroku związanego z metodą potęgową oraz deflacja po znalezieniu $\lambda_i$ w celu znalezienia wszystkich wartości własnych macierzy $A$ (funkcja \texttt{findEigenvaluesAndVectors})
		\item Obliczenie macierzy błędu $E$ (funkcja \texttt{calculateErrorMatrix})
	\end{enumerate}

	
	
	\section{Poprawność metody}
	Dla przykładowych punktów pomiarowych testowanych przeze mnie metoda była poprawna, to jest znajdowała funkcję, która aproksymowała te punkty. Zdarzało się jednak, że aproksymacja ta nie była zbyt dokładna.
	
	Zatem ogólnie można stwierdzić, że metoda jest poprawna. W swoich obliczeniach w sekcji 2 nie stosowałem żadnych restrykcyjnych założeń, zatem powinna być wykonalna dla wszystkich zestawów punktów pomiarowych, z zastrzeżeniem, że dla niektórych może nie być dokładna.
	
	
	
	
	\section{Przykłady}
	Przykład 1 został wygenerowany poprzez obliczenie danej funkcji aproksymowanej $f$ w węzłach, a następnie zaburzenie tych wartości poprzez dodanie szumu. Dopiero na tych zaburzonych wartościach wykonano aproksymację.
	
	W przykładach 2-4 zostały użyte dokładne wartości w węzłach dla funkcji aproksymowanej $f$.
	
	W przykładach 5-6 użyte zostały wartości pomiarowe, nie mające powiązania z żadną funkcją.
	
	
	\begin{enumerate}[label=\textbf{Przykład \arabic*}]
		\item
		Funkcja aproksymowana: $f(x) = x^2$.\\
		Zaburzenie wartości:
		\begin{itemize}
			\item mnożnik z zakresu $<\frac{1}{2}, 2>$
			\item dodanie stałej z zakresu $<0, 10>$
		\end{itemize}
		Węzły: $x \in \{-1, 1, 2, 4, 5, 6\}$

	
		Dokładność aproksymacji zależy głównie od wylosowanego szumu. Element optymalny dość dobrze aproksymuje węzły, jednakże ze względu na duże zaburzenie wartości w węzłach nie aproksymuje on dobrze funkcji wejściowej $f(x) = x^2$.
		
		\item
		Funkcja aproksymowana: $f(x) = x^2$.\\
		Brak zaburzenia wartości.\\
		Węzły: $x \in \{-1, 1, 2, 4, 5, 6\}$
		
		Metoda dokładnie aproksymuje funkcję z bazy przestrzeni elementu optymalnego, jeżeli dane nie są zaburzone.
		
		\item
		Funkcja aproksymowana: $f(x) = x^3$.\\
		Brak zaburzenia wartości.\\
		Węzły: $x \in \{-1, 1, 2, 4, 5, 6\}$
		
		Metoda dobrze aproksymuje funkcję w okolicy węzłów, dla $x < -1$ błąd aproksymacji jest coraz większy.
		
		\item
		Funkcja aproksymowana: $f(x) = \cos x + \tan x + x^3$.\\
		Brak zaburzenia wartości.\\
		Węzły: $x \in \{-\frac{\pi}{3} -\frac{\pi}{4}, \frac{\pi}{6}, 0, \frac{\pi}{6}, \frac{\pi}{4}, \frac{\pi}{3} \}$
		
		Dla funkcji składającej się z sumy funkcji spoza bazy elementu optymalnego metoda wykonała dość dobrą aproksymację w przedziale, na którym występują węzły. Poza tym przedziałem ($|x| > \frac{\pi}{3}$) błąd aproksymacji wzrasta.
		
		\item
		Węzły:
		\begin{align*}
			\begin{tabular}{|c|c|c|}
				\hline
				i & $x_i$ & $f_i$ \\ \hline
				1 &  1  & 40  \\ \hline
				2 &  2  & 10  \\ \hline
				3 &  3  & -4  \\ \hline
				4 &  4  & 10  \\ \hline
			\end{tabular}
		\end{align*}
		
		
		Gdy liczba węzłów jest równa liczbie funkcji z bazy (4), to aproksymacja staje się interpolacją, ponieważ w węzłach wartości elementu optymalnego są równe wartościom pomiarowym.
		
		\item
		Węzły:
		\begin{align*}
			\begin{tabular}{|c|c|c|}
				\hline
				i & $x_i$ & $f_i$ \\ \hline
				1 &   1   &  50   \\ \hline
				2 &   2   &  -20  \\ \hline
				3 &   3   &  18   \\ \hline
				4 &   4   &  -25  \\ \hline
				5 &   5   &  10   \\ \hline
				6 &   6   &  70   \\ \hline
				7 &   7   &  -10  \\ \hline
			\end{tabular}
		\end{align*}
		

		W przypadku dużych zmianach między wartościami kolejnych węzłów aproksymacja nie jest dokładna.
		
	\end{enumerate}
	
	
	
	
	\section{Wnioski}
	\begin{enumerate}
		\item Element optymalny w tym przypadku odwzorowuje dokładnie funkcje z bazy (gdy dane nie są zaburzone) (przykład 2).
		\item Dla 4 punktów pomiarowych aproksymacja jest dokładna (staje się interpolacją). Dla większej ilości punktów nie ma takiej zależności (przykład 5).
		\item Im więcej punktów pomiarowych, pomiędzy którymi wstępują większe moduły zmian wartości, tym aproksymacja jest gorsza (przykład 6).
		\item Aproksymacja jest zazwyczaj dość dobra w przedziale, na którym występują węzły. Poza nim, jak można się zresztą spodziewać, błąd aproksymacji wzrasta (przykłady 3, 4).
	\end{enumerate}
	
	
	
	\section{Funkcja do testowania metody}
	Funkcja \texttt{plotApproximation(x, f, pointsCount, paddingMultiplier)} umożliwia łatwe testowanie metody. Oblicza ona element optymalny, a następnie wyświetla wykres elementu optymalnego oraz nanosi na niego punkty pomiarowe.
	
	Obszar, na jakim rysowany jest wykres jest powiększony o $paddingMultiplier * (\max_{i \in {1, \dots, m}} x_i - \min_{i \in {1, \dots, m}} x_i)$ w celu zobrazowania elementu optymalnego również poza zakresem punktów pomiarowych. Domyślna wartość $paddingMultiplier = \frac{1}{6}$.
	
	Argumenty przyjmowane przez tą funkcję opisane są w pliku \textit{plotApproximation.m} (nad nagłówkiem funkcji).
	
	Przykładowe wywołanie funkcji:
	\texttt{plotApproximation([0 1 2 3], [5 6 8 -5])}
	
	Dodatkowo aby ułatwić sprawdzanie tej metody aproksymacji dla funkcji został stworzony skrypt \texttt{approximateFunction.m}. Zawiera on obliczenie wartości danej funkcji $f$, a następnie dodanie \textit{szumu} do każdego węzła aproksymacji:
	$$
	f_i = rand(S_{min}, S_{max}) +  f(x_i) * rand(P_{min}, P_{max})
	$$
	gdzie $S_{min}, S_{max} \in \mathbb{R}$ są stałymi, które zostaną dodane do każdej obliczonej wartości funkcji, $P_{min}, P_{max} \in \mathbb{R}$ oznaczają zakres procentowy przez który zostaną pomnożone obliczone wartości funkcji.
	
	Dzięki temu można faktycznie sprawdzić poprawność aproksymacji dla zaburzonych danych.
	
	
	\section{Interfejs graficzny}
	Do metody został dołączony interfejs graficzny umożliwiający łatwe wprowadzanie punktów pomiarowych, zmianę parametrów (współczynnik powiększenia obszaru, a także ilość punktów wewnątrz obszaru, na których zostanie narysowany wykres) i dokonanie z tymi danymi aproksymacji.
	
	
	
	\section{Bibliografia}
	\begin{enumerate}
		\item Informacje z wykładu \textit{Metod numerycznych 2} (wydział MiNI PW, dr Iwona Wróbel), w szczególności temat dotyczący \textit{metody potęgowej z normowaniem}, wraz z algorytmem.
	\end{enumerate}
	
\end{document}